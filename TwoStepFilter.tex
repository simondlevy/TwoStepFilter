\documentclass[10pt,journal,compsoc]{IEEEtran}
\usepackage{amsmath,bm,mathtools}
\usepackage{tabularx,multirow,booktabs,blindtext}
\usepackage[T1]{fontenc}
\ifCLASSOPTIONcompsoc
\usepackage[nocompress]{cite}
\else
% normal IEEE
\usepackage{cite}
\fi

% correct bad hyphenation here
\hyphenation{op-tical net-works semi-conduc-tor}

\begin{document}

\title{A Two-step Kalman/Complementary Filter for Estimation of Vertical Position
Using an IMU-Barometer System}

\author{Jung Keun Lee
\IEEEcompsocitemizethanks{\IEEEcompsocthanksitem Department of Mechanical Engineering, Hankyong
National University
327 Jungang-ro, Anseong, Gyeonggi, 17579, Korea\protect\\
Corresponding author: jklee@hknu.ac.kr}
\thanks{(Received: April 15, 2016; Accepted : May 30, 2016)
This is an Open Access article distributed under the terms of the Creative
Commons Attribution Non-Commercial License(http://creativecommons.org/
licenses/bync/3.0) which permits unrestricted non-commercial use, distribution,
and reproduction in any medium, provided the original work is properly cited.}
\footnotesize \\Translated from Korean by Simon D. Levy (simon.d.levy@gmail.com)}\normalsize

% The paper headers
\markboth{Journal of Sensor Science and Technology 
Vol. 25, No. 3 (2016) pp. 202-207
http://dx.doi.org/10.5369/JSST.2016.25.3.202
SSN 1225-5475/eISSN 2093-7563}\%

\IEEEtitleabstractindextext{%
\begin{abstract}
Estimation of vertical position is critical in applications of sports science
and fall detection and also controls of unmanned aerial vehicles and
motor boats. Due to low accuracy of GPS(global positioning system) in the
vertical direction, the integration of IMU(inertial measurement unit) with
the GPS is not suitable for the vertical position estimation. This paper
investigates an IMU-barometer integration for estimation of vertical
position (as well as vertical velocity). In particular, a new two-step
Kalman/complementary filter is proposed for accurate and efficient
estimation using six-axis IMU and barometer signals. The two-step filter is
composed of (i) a Kalman filter that estimates vertical acceleration via
tilt orientation of the sensor using the IMU signals and (ii) a
complementary filter that estimates vertical position using the barometer
signal and the vertical acceleration from the first step. The estimation
performance was evaluated against a reference optical motion capture
system. In the experimental results, the averaged estimation error of the
method was 19.7 cm while that of the raw barometer signal was 43.4
cm.  
\end{abstract}

\begin{IEEEkeywords}
Vertical position, Vertical acceleration, Kalman filter, Complementary filter, IMU(inertial measurement unit), Barometer
\end{IEEEkeywords}}


% make the title area
\maketitle


\IEEEdisplaynontitleabstractindextext
\IEEEpeerreviewmaketitle

\IEEEraisesectionheading{\section{Introduction}\label{sec:introduction}}

\IEEEPARstart{A}{ccurate} vertical position estimation for moving objects or
humans is required in various fields. For example, in the control of an
unmanned aerial vehicle (UAV) such as a drone, altitude may be considered as a
kind of vertical displacement [1].  In sports such as skiing or
snowboarding where the vertical displacement is severe, vertical displacement
estimation using a portable sensor system can be analyzed to improve
performance [2].

In order to overcome the limitations of most motion-capture systems in
tracking trajectories of moving objects, fusion of GPS (Global Positioning
System) and IMU (Inertial Measurement Unit) has been attempted.  At this time,
GPS provides displacement values that do not drift, and inertial sensors
provide high sampling rates, so that high sampling rate and high accuracy
displacement estimation are achieved through fusion of the two sensors.
However, the error of vertical direction displacement of GPS is about 10 --
20m, which is much lower than that of horizontal displacement [3]. 

As an alternative to this, a barometer can be utilized for vertical
displacement. However, a barometer is very noisy when used alone, because
it estimates vertical displacement through the sensing of the
atmospheric pressure change. It responds sensitively to atmospheric conditions,
indoor / outdoor conditions, and even the degree of window opening, all of
which cause errors in the calculation of vertical displacement. Thus, similar
to IMU-GPS fusion, barometric-IMU convergence has been studied for
high-sampling rate and high-accuracy vertical displacement estimation [2,4-7].
In particular, IMU-barometer fusion is used in pedestrian navigation [8]
and fall detection [9], two varieties of human monitoring.

Two approaches to IMU / barometer fusion can be considered: tightly
coupled and loosely coupled [5]. The tightly-coupled method is effective in
modeling the noise of the two signals in a way that the signals of the IMU and
the barometer are fused from the beginning of the filter, but the system matrix
is large, as is the amount of computation required.  The loosely-coupled method
employs a two-step filter and is used more frequently because of convenience
of application and efficiency of computation. In this approach, the two-stage
filter computes the vertical displacement using (i) the first filter to
calculate the attitude of the sensor using the IMU signal and obtain the
vertical acceleration through it, and (ii) the second filter to fuse the
vertical acceleration calculated from the barometer signal with the output of
the previous filter.

Zihajehzadeh et al. [2] proposed a Kalman filter (KF) [10], which uses a six-axis
IMU developed by the authors (i.e., a three-axis accelerometer and a three-axis
gyroscope). Another Kalman filter, which sets the vertical displacement and
the vertical velocity as state variables, was used as the second filter.
Tanigawa et al. [7] applied the same Kalman filter as the first filter to the
Xsens three-dimensional posture calculation based on a nine-axis IMU (i.e., six-axis
IMU + three-axis geomagnetic sensor). Sabatini and Genovese [5]
proposed an Extended Kalman Filter (EFK) that obtains quaternions
using a six-axis IMU for the first filter and a complementary filter (CF) as the second
filter. In the EKF proposed by Son and Oh [4], the IMU can be calibrated during
measurement by adding the accelerometer bias and scale factor in addition
to the vertical displacement and vertical velocity as state variables, thereby
improving the accuracy of posture and vertical displacement estimation.

Among the above methods, all except [5] use a Kalman filter as the second filter,
to exploit the Kalamn Filter's optimization and smoothing effects.  However,
the Kalman filter has a disadvantage that the computation required is larger than
that of the complementary filter.

In this paper, a two-stage Kalman / complementary filter is proposed.  Accurate
vertical acceleration is estimated by using the tilt-estimation Kalman filter
[10] adopted in [2] as the first filter. For the second filter, we propose a new
combination IMU-barometric-based Kalman / complementary filter using the
highly efficient complementary filter introduced in [11].  In addition, we
examine (1) the effect of the accuracy of vertical-acceleration estimation on
the accuracy of vertical-displacement estimation, and (2) the comparison
between short and long endpoints and estimation accuracy based on the selection of
complementary filter and Kalman filter in the second stage. In sum,
we propose an optimal vertical displacement estimation filter that combines 
accuracy of estimation with computational efficiency.

\section{Estimation algorithm and verification experiment}

\subsection{Kalman filter for vertical acceleration estimation via posture estimation}

The Kalman filter for vertical acceleration estimation, which is the first
step, estimates the tilt as a vertical axis slope using the accelerometer and
gyroscope signals from a six-axis IMU [10], and compensates for the
gravitational acceleration component in the accelerometer signal (See Fig. 1).

The signals of the gyroscope (G) and the accelerometer (A) are modeled as
follows:

\[\bm{s}_G = \prescript{S}{}{\bm{\omega}}+\bm{n}_G\tag{1.a}\]

\[\bm{s}_A = \prescript{S}{}{\bm{g}}+\prescript{S}{}{\bm{a}}+\bm{n}_G\tag{1.b}\]


\noindent where $\bm{\omega}$ is the angular velocity, $\bm{a}$ is the sensor
acceleration, and $\bm{n}$ is the measurement noise. The superscript $S$ 
means that the vector is represented in the sensor coordinate system. In
equation (1.b), the sensor acceleration is modeled as a first-order Markov
chain process:

\[\prescript{S}{}{\bm{a}_t} = c_a\prescript{S}{}{\bm{a}_{t-1}}+\bm{\varepsilon}_{a,t}\tag{2}\]

\noindent where $c_a$ and $\bm{\varepsilon}_{a,t}$ are a constant parameter and
the time-varying error term of the acceleration model, respectively.

The first filter estimates the tilt attitude expressed by
$\prescript{S}{}{\bm{Z}}$ as a state vector, and obtains the vertical
acceleration through the estimation. Here $\prescript{S}{}{\bm{Z}}$ is a
representation of the $Z$-axis unit vector of the inertial coordinate system (I) in
the sensor coordinate system.  It is part of a direction cosine matrix which is
a three-dimensional attitude matrix. First, the process model that updates the
state variable $x_1 (= \prescript{S}{}{\bm{Z}})$ over time is expressed as
follows from the strapdown integration associated with the angular velocity
measurement of the gyroscope:

\begin{figure}[!t]
\centering
\includegraphics[width=2.5in]{fig1}
    \caption{Flowchart of the proposed two-step Kalman/complementary filter.}
\label{fig1}
\end{figure}


\[\bm{x}_{1,t}=\prescript{S}{}{\bm{Z}_t}=
(\bm{I}-\Delta t\bm{\tilde{s}}_{G,t-1})\prescript{S}{}{\bm{Z}_{t-1}}+
\Delta t(-\prescript{S}{}{\bm{\tilde{Z}}_{t-1}})\bm{n}_G\tag{3}\]

Here, $\Delta t$ is the sampling interval and the tilde ($\tilde{~}$) denotes the outer matrix of
the vector, e.g., $\tilde{a} = [a \times]$.

The measurement model is a mixture of accelerometer signal and sensor acceleration model as follows:

\[\bm{s}_{A,t}-c_a\prescript{S}{}{\bm{a}^+_{t-1}}=g\prescript{S}{}{\bm{Z_t}}\prescript{S}{}{\bm{a}^-_{\varepsilon,t}}+\bm{n}_A\tag{4}\]


The following relations hold in the above equation: 
$\prescript{S}{}{\bm{g}} = g\prescript{S}{}{\bm{Z}}$, 
$\prescript{S}{}{\bm{a}^-_{\varepsilon,t}} = \prescript{S}{}{\bm{a}^-_t} - \prescript{S}{}{\bm{a}_t}$, 
The superscripts $-$ and + mean the prior and the posterior, respectively.

From equations (3) and (4) we obtain the following KF equation:

\[\bm{x}_{1,t} = \bm{\Phi}_{t-1}\bm{x}_{1,t-1} + \bm{w}_{t-1}\tag{5.a}\]

\[\bm{z}_t = \bm{H}_t \bm{x}_{1,t-1} + \bm{v}_t\tag{5.b}\]


\noindent where the transition matrix $\bm{\Phi}_{t-1}$ is $I - \Delta t
\bm{\tilde{s}}_{G,t-1}$; the process noise $\bm{w}_{t-1}$ is $\Delta t (-
\prescript{S}{}{\bm{\tilde{Z}}_{t-1}}) \bm{n}_G$; 
the measurement vector $\bm{z}_t$ is $\bm{s}_{A, t} - c_a \prescript{S}{}{\bm{a}}^+_{t-1}$;
The observation matrix $\bm{H}_t$ is $g\bm{I$}; and the measurement noise $\bm{v}_t$ is 
$-\prescript{S}{}{\bm{a}}_{\varepsilon,t} + \bm{n}_A$. The covariance matrix, 
$\bm{Q}_{t-1}$ (= $E [\bm{w}_{t-1} - \bm{w}^T_{t-1}]$) and $\bm{M}_t$ (= $E [\bm{v}_t \bm{v}^T_t]$) 
for the process noise and the measurement noise are as follows:

\[\bm{Q}_{t-1} = -\Delta t^{2}\prescript{S}{}{\tilde{\bm{Z}}_{t-1}\Sigma_G}\prescript{S}{}{\tilde{\bm{Z}}_{t-1}}\tag{6.a}\]

\[\bm{M}_t = \Sigma_{acc} + \Sigma_A\tag{6.b}\]

\noindent where $E$ is the expectation operator, the covariance matrix $\Sigma_G$ for gyro
measurement noise is $\sigma^2_G I_3$; the covariance matrix $\Sigma_A$ for accelerometer
measurement noise is set to $\sigma_A \textbf{I}_3$; and $\sigma_G$ and $\sigma_A$ are noise standard
deviations. The covariance matrix $\Sigma_{acc}$ of the acceleration model error defined
by $E((\prescript{S}{}{\bm{a}}^+_{\varepsilon,t})(\prescript{S}{}{\bm{a}}^+_{\varepsilon,t})^T)$ 
is set to $3^{-1}c^2_a||\prescript{S}{}{\bm{a}}^+_{t-1}||\textbf{I}$.

Once $\prescript{S}{}{\bm{Z}}$ is obtained, the external acceleration
$\prescript{S}{}{\bm{a}}$ from the viewpoint of the sensor coordinate system is
obtained as  $\bm{s}_A-g\prescript{S}{}{\bm{Z}}$, and finally the $\bm{Z}$-direction
acceleration $\prescript{I}{}{a}_z$ from the viewpoint of the inertial coordinate system is
obtained by the following equation:

\[a_z(=\prescript{I}{}{a}_z)= \prescript{S}{}{\bm{a}}^T \prescript{S}{}{\bm{Z}}\tag{7}\]


\subsection{Complementary filter for vertical displacement estimation}

In the second step, a complementary filter, the vertical displacement 
$h_z (= \prescript{I}{}{h}_z)$ and the vertical velocity $v_z (= \prescript{I}{}{v}_z)$ 
are estimated using the vertical acceleration transmitted through the first Kalman filter and the barometric
signal.

The barometer pressure $P$ can be converted to the vertical displacement $h_z$ by
the following equation [12].

\[h_z = 44330(1-(P/P_0)^{0.19})-h_{init}\tag{8}\]

\noindent where the unit of $h_z$ is m(eters) and $P_0$ is the atmospheric pressure at sea level
= 101,352 Pa(scals), and $h_{init}$ is the altitude of the starting point of
observation. In other words, $h_z$ in this paper is the variation of the vertical
displacement from the observation starting point.  Also, in this paper, based
on the highly transformed barometric pressure signal, the signal of the
barometer ($B$) is modeled as

\[s_B = h_z + n_B\tag{9}\]

The state vector $\bm{x}_2$ in the second (complementary) filter is [11]:

\[\bm{x}_2 = [h_z v_z]^T\tag{10}\]

The complementary filter is applied as follows:

\begin{equation}
\resizebox{.42 \textwidth}{!} 
{
$ 
\bm{x}_{2,t} = 
\begin{bmatrix} 1 & \Delta t \\ 0 & 1 \end{bmatrix} \bm{x}_{2,t-1} +
\begin{bmatrix} 1 & \Delta t / 2 \\ 0 & 1 \end{bmatrix} \bm{K}_c\Delta t \times \varepsilon_{h,t-1} +
\begin{bmatrix} \Delta t / 2 \\ 1 \end{bmatrix} \Delta v_{z,t-1}
$
}
\tag{11}
\end{equation}

Here, $\varepsilon_h$ is the difference between the barometric signal and the estimated value:

\[\varepsilon_{h,t-1} = s_{B,t-1} - h_{z,t-1}\tag{12}\]

\noindent
and the velocity increment $\Delta v_z$ is $\Delta t \times a_Z$. In addition, the complementary filter gain 
$\bm{K}_c$ is as follows:

\[\bm{K}_c = \begin{bmatrix}  \sqrt{2 (\sigma_{acc}/\sigma_B)} \\ \sigma_{acc}/\sigma_B \end{bmatrix} \tag{13}\]

\noindent where $\sigma_{acc}$ and $\sigma_B$ are the vertical acceleration estimate and the barometric signal,
respectively.   For reference, this complementary filter has low-pass filter
characteristics of time constant $\tau = \sqrt{\sigma_B/\sigma_{acc}}$.


Prior to the above, the zero-velocity update (ZUPT) was applied.  Based
on the integration of noise, this is a technique to limit the drift error. When the
zero speed is detected (as follows), the speed is set to zero instead of being set to
the complementary filter integral from Eq. (11):

\[v_{z,t} = \begin{cases} 0, if (|a_{z,\tau}|<0.1m/s^2 \forall_\tau\in[t-n\Delta t, t]) \\ Eq. (11)~otherwise \end{cases} \tag{14}\]

\noindent where $n$ is set to 12.

The overall configuration of the proposed two-stage Kalman / complementary filter is shown in Fig. 1.

\begin{figure}[!t]
\centering
\includegraphics[width=2.5in]{fig2}
\caption{GY-87 IMU-Barometer and Arduino board system.}
\label{fig2}
\end{figure}


\subsection{Verification experiment}

A GY-87 modular system was used for verification experiments. The GY-87
consists of a six-axis InvenSense MPU-6050 IMU (including accelerometer and
gyro), a three-axis Honeywell HMC5883L geomagnetic sensor, and a Bosch BMP180
barometer, of which IMU and barometer were used for this experiment. (Refer to
Table 1 for details.)  The setup of the GY-87 is shown in Fig. 2 with an Arduino
Uno R3 microcontroller and USB serial communication with the
PC.  The OptiTrack Flex13 optical motion capture system
(NaturalPoint, Inc. USA) of Figure 3 was used as a ground-truth reference to obtain
root mean square error (RMSE).  

In addition to the proposed two-stage Kalman / complementary filter method,
the following cases were compared and analyzed:

\begin{itemize}
\item The method of obtaining vertical acceleration: 
\begin{enumerate}
\item $a_{z,KF}$: the proposed method described in Section 2.1 using both accelerometer and gyroscope,
\item $a_{z,OPT}$: a method of obtaining vertical acceleration using the highly
accurate posture obtained from the Flex 13 camera system instead of the inertial sensor,
\item $a_{z,app}$: Approximate method using only accelerometer [13]. That is, $a_{z,app} = ||s_A|| - g$.
\end{enumerate}
\item How to find vertical displacement:
\begin{enumerate}
\item Using the complementary filter described in Section 2.2,
\item $[2]$ and [7] using Kalman filter.
\end{enumerate}
\end{itemize}

Five experiments were conducted as follows. As shown in Fig. 3., Tests A to C 
were performed in the indoor gymnasium, elevated above the
starting height by more than 3m. The optical marker was used only for the
reference value corresponding to the vertical displacement (That is, $a_{z,OPT}$
was not obtained through the reference value of the sensor attitude.) 
For tests D and E on the other hand, the apparatus was lowered to within 1.5 m from the starting height.
To confirm the influence of $a_{z,KF}$ and $a_{z,OPT}$ on the vertical displacement
estimation, three additional markers were attached.  The test movements were as follows:
\begin{itemize}
    \item \textbf{Test A} Increased gradually from the starting height to around 3.5m.
         Waited for about 10 seconds before descending to the starting height again.
         During these upward and downward motions, the sensor was also tilted from side to side
         by at least 45 degrees.  This experiment tested the influence of vertical
         acceleration $a_{z,KF}$ and attitude change $a_{z,app}$ on the estimation of
         vertical displacement.
    \item \textbf{Test B-C} Repeatedly ascending and descending by up to about 3m. Compared to Test B, 
        Test C had faster and more frequent ups and downs. This experiment
        tested the influence of movement speed on vertical displacement estimation.
    \item \textbf{Test D} Cascaded steps up and down, then stop.
    \item \textbf{Test E} Observe the transition of the barometer signal at small altitude changes.
\end{itemize}

\begin{figure}[!t]
\centering
\includegraphics[width=2.5in]{fig3}
\caption{Test setup with the Fex 13 optical motion capture system in an indoor gym.}
\label{fig3}
\end{figure}


\begin{figure}[!t]
\centering
\includegraphics[width=2.5in]{fig4}
\caption{Comparison of vertical position estimations.}
\label{fig4}
\end{figure}


\section{Results}

Table 2 shows the RMSE for each case, and Fig. 4 shows the result of applying
the CF in the second step.

Test A - C results: effect of applying $a_{z,app}$. In cases of vertical
motion in which the change of posture was not large (Test B or Test C),
since $a_{z,app}$ maintains some accuracy, the vertical displacement estimate
was also obtained more accurately than with the barometer by itself ($s_B$).
However, in Test A, when the sensor posture changed significantly and
vertical motion was performed, the estimation accuracy was even worse
than the RMSE of $s_B$. The maximum error was 26.2 cm in Test C, and
even less than 30 cm of accuracy was achieved in a displacement test of 3 m or
more. In Test C, it seems that the influence of the large error of the
barometer signal is reflected in the estimated value.

Test D - E results: Overall accuracy was highest for $a_{z,OPT}$, which estimated
vertical displacement based on very accurate vertical acceleration. However,
the proposed $a_{z,KF}$ method also showed very high accuracy,
averaging 1.4 cm over $a_{z,OPT}$ method.  The $a_{z,app}$ method showed very
different results depending on the degree of sensor posture maintained.  For
example, in Test D, the RMSE was less than 20 cm for a motion
that included a stationary state with no change in posture.

The barometer signal functions to prevent drift errors and ultimately
determines the direction of the estimated value. In other words, it is
inevitable that the estimated value is also affected if the error increases because
of an abnormal barometer signal.  For example, looking at the portion after
35 seconds in Test E, the sensor stopped at the 0 m initial position, but the
barometer signal continued to drop, and you can see how the estimated value
tracks it.  Therefore, in order to improve estimation performance, it is
necessary to improve the performance of the barometer itself.

In conclusion, there was no performance difference between CF and KF in all
five tests.  However, considering the amount of calculation (CF / KF
calculation ratio = 12.5\%), it can be concluded that CF is advantageous
overall. The proposed method has an average RMSE of 19.7 cm, which is 55\%
better than the RMSE 43.4 cm of the barometer signal by itself.  In [2], which
estimated vertical acceleration with a Kalman filter and vertical displacement
with a Kalman filter (i.e., a two-stage Kalman / Kalman filter), an average RMSE
27.4 cm was obtained.

\section{Review and Discussion}

In this paper, a two-stage Kalman / complementary filter for vertical
displacement estimation based on IMU-barometer fusion was proposed. The first
stage uses a Kalman filter (KF) to estimate vertical acceleration
via the tilt posture of a six-axis IMU signal. The second stage
uses a complementary filter (CF) to estimate vertical displacement and vertical
velocity using vertical acceleration and a barometric signal
transmitted through the first Kalman filter. The following conclusions are
drawn from various experiments.

1. In the vertical displacement estimation filter using the vertical
acceleration estimate and the barometric signal, there was almost no difference
in accuracy between KF and CF (absolute difference value average 0.15 cm).
However, considering the calculation time, the CF method is eight times faster than KF
so we conclude that the CF adopted by this paper is better.

2. Unless there is little change in attitude, approximated vertical
acceleration estimation using only an accelerometer is inadequate for use in
estimating vertical displacement (despite the ease of sensor configuration and
method).

3. In the vertical acceleration estimation based on the IMU-barometric fusion
sensor, the performance of the barometer has a great influence on the
estimation accuracy, and it seems that there is a limit to improving the accuracy
through the IMU.  However, in comparison with the barometer signal error of
43.4cm average RMSE, the average RMSE error of the proposed method was 19.7cm.

The method proposed in this paper -- efficiently improving the accuracy of
vertical displacement estimation through IMU-barometric fusion -- can be
widely applied to sports science and ship VDR (voyage data recording), as well as
pedestrian navigation and fall detection.

\section*{Acknowledgments}

The work described in this paper was carried out with support from the Small
and Medium Business Administration's Research Village project (C0301478) and
the Korea Research Foundation (NRF-2015R1C1A1A02036373) of the Institute for
Future Creation Sciences.

\begin{table}[tp]
\caption{Specification of GYU-87 IMU-Barometer system.}
\vspace*{-5mm}
\label{table:levels}
\centering
\begin{tabular}{llll}
\\ \toprule
& Accelerometer & Gyroscope & Barometer\\ \midrule

Model 
& \multicolumn{1}{ p{1.5cm} }{\centering Invensense \\ MPU-6050} 
& \multicolumn{1}{ p{1.5cm} }{\centering Invensense \\ MPU-6050} 
& \multicolumn{1}{ p{1.5cm} }{\centering Bosch \\ BMP 180} \\
\hline\\

Full-Scale Range 
& $\pm 2 \sim 16 g$ 
& \multicolumn{1}{ p{1.5cm} }{\centering $\pm 250 \sim$ \\ 2000 deg/s} 
& \multicolumn{1}{ p{1.5cm} }{\centering $\pm 300 \sim$ \\ 1100 hPa} \\
\hline\\

Sensitivity
& \multicolumn{1}{ p{1.5cm} }{\centering $0.000061 \sim$ \\ 0.0049 g} 
& \multicolumn{1}{ p{1.5cm} }{\centering $0.0076 \sim$ \\ 0.061 deg/s} 
& \multicolumn{1}{ p{1.5cm} }{\centering 0.0015 hPa} \\
\hline\\

\multicolumn{1}{ p{2cm} }{\centering Max. Sampling\\ Rate} 
& 1000 Hz
& 8000 Hz
& 128 Hz \\
\hline\\

Digital Resolution
& 16 bit
& 16 bit
& 19 bit\\
\hline

\end{tabular}
\end{table}  

\renewcommand{\arraystretch}{1.5}

\begin{table}[tp]
    \caption{RMSE (root mean squared error) of raw barometer signal and estimations of each case
    (unit: cm)}
\vspace*{-5mm}
\label{table:rmse}
\centering
\begin{tabular}{llllll}
\\ \toprule
\textbf{Test} & $S_B$ & \textbf{Filter} & $a_{z,OPT}$ & $a_{z,KF}$ & $a_{z,app}$ \\ \midrule
Test A & 41.7 & CF & N/A & 19.9 & 86.8\\ \hline &  & KF & N/A & 19.8 & 86.9\\ \hline
Test B & 42.0 & CF & N/A & 13.9 & 37.4\\ \hline &  & KF & N/A & 13.8 & 37.4\\ \hline
Test C & 50.4 & CF & N/A & 26.2 & 32.4\\ \hline &  & KF & N/A & 26.2 & 32.4\\ \hline
Test D & 40.8 & CF & 16.8 & 18.5 & 18.2 \\ \hline &  & KF & 16.3 & 18.1 & 18.6\\ \hline
Test E & 42.2 & CF & 18.7 & 19.9 & 32.1 \\ \hline &  & KF & 18.7 & 19.8 & 32.0\\ \hline
Average & 43.4 & & 17.8 & 19.7 & 41.4 \\ \hline &  & & 17.5 & 19.5 & 41.5\\ \hline
\\
\\
\end{tabular}
\end{table}  

\begin{thebibliography}{1}

    \bibitem{Gasior}
        P. G\k{a}sior, S. Gardecki, J. Go\'{s}li\'{n}ski, and W. Giernacki,
        ``Estimation of altitude and vertical velocity for multirotor
        aerial vehicle using Kalman filter,'' in \emph{Recent Advances in
        Automation, Robotics and Measuring Techniques}, Vol. 267,
        R. Szewczyk, C. Zieli\'{n}ski, and M. Kaliczy\'{n}ska, Eds. Hei-
        delberg, Germany: Springer International Publishing, 2014,
        pp. 377-385.

    \bibitem{Zihajehzadeh}
        S. Zihajehzadeh, T. J. Lee, J. K. Lee, R. Hoskinson, and E.
        J. Park, ``Integration of MEMS inertial and pressure sensors
        for vertical trajectory determination,'' \emph{IEEE Trans. Instrum.
        Meas.}, Vol. 64, No. 3, pp. 804-814, 2015.

    \bibitem{Svbavensky}
        O. \v{S}v\'{a}bensk\'{y}, J. Weigel, and R. Machotka, ''On GPS
        heighting in local networks,'' \emph{Acta Geodyn. Geomater.}, Vol.
        3, No. 143, pp. 39-43,Jun. 2006.

    \bibitem{Son}
        Y. B. Son, and S. Y. Oh, ``A barometer-IMU fusion method
        for vertical velocity and height estimation,'' \emph{IEEE Proc. of
        Sensors}, pp. 1-4, Busan, Korea, 2015.

    \bibitem{Sabatini}
        A. M. Sabatini and V. Genovese, ``A sensor fusion method
        for tracking vertical velocity and height based on inertial
        and barometric altimeter measurements,'' \emph{Sensors}, Vol. 14,
        No. 8, pp. 13324-13347, 2014.

    \bibitem{Pierleoni}
        P. Pierleoni, A. Belli, L. Palma, L. Pernini, and S. Valenti,
        ``An accurate device for real-time altitude estimaion using
        data fusion algorithms,'' \emph{Proc. of 2014 IEEE/ASME 10th
        Int'l Conf. on MESA.}, pp.1-5, Senigallia, Italy, 2014.

    \bibitem{Tanigawa}
        M. Tanigawa, H. Luinge, L. Schipper, P. Slycke, ``Drift-free
        dynamic height sensor using MEMS IMU aided by MEMS
        pressure sensor,'' \emph{Proc. 5th Workshop on Positioning, Nav-
        igation and Communication}, pp. 191-196, Hannover, Ger-
        many, 2008.

    \bibitem{Kim}
        Y. Kim, Y. Hwang, S. Choi and J. Lee, ``Height estimation
        scheme of low-cost pedestrian dead-reckoning system using
        Kalman filter and walk condition estimation algorithm,''
        \emph{Proc. of IEEE/ASME International Conf. on AIM.}, pp.1492-
        1497, Wollongong, Australia, 2013.

    \bibitem{Bianchi}
        F. Bianchi, S. J. Redmond, M. R. Narayanan, S. Cerutti, and
        N. H. Lovell, ``Barometric pressure and triaxial acceler-
        ometry-based falls event detection,'' \emph{IEEE Trans. Neural
        Sys. Rehab. Eng.}, Vol. 18, No. 6, pp. 619-627, 2010.

    \bibitem{Lee}
         J. K. Lee, E. J. Park, and S. N. Robinovitch, ``Estimation of
        attitude and external acceleration using inertial sensor mea-
        surement during various dynamic conditions'' \emph{IEEE Trans.
        Instrum. Meas.}, Vol. 61, No. 8, pp. 2262-2273, 2012.

    \bibitem{Walter}
         T. Walter, and J. R, Higgins, ``A comparison of comple-
        mentary and Kalman filtering,'' \emph{IEEE Trans. Aerosp. Elec-
        tron. Syst.}, Vol. AES-11, No. 3, pp. 321-325, 1975.

    \bibitem{Sabatini}
         A. M. Sabatini and V. Genovese, ``A stochastic approach to
        noise modeling for barometric altimeters,'' \emph{Sensors}, Vol. 13,
        No. 11, pp. 15692-15707, 2013.

    \bibitem{Degen}
         T. Degen, H. Jaeckel, M. Rufer, and S. Wyss, ``SPEEDY: a
        fall detector in a wrist watch'' in \emph{Proc. 7th IEEE Int. Sym-
        posium on Wearable Computing}, pp. 184-187, 2003.

    \end{thebibliography}


\end{document}
